 
\selectlanguage{spanish}

\begin{abstract}
 Actualmente el la revolución de fuentes energéticas y los cambios acelerados 
 de las condiciones climáticas obliga a las naciones a plantear un cambio a los 
 enfoques de generación de energía, para esta serie de cambios se requiere un 
 análisis de todas y cada una de las fuentes de energía renovables y en varias 
 situaciones es pertinente resaltar las condiciones necesarias para optimizar el aprovechamiento del potencial energético. 
 La actual investigación se centró en la  construcción de una serie de tiempo en radiación 
 solar presente en el departamento de Nariño, con el objetivo principal de brindar 
 apoyo en el estudio de indicadores para radiación solar en el proyecto “Análisis de oportunidades energéticas con 
 fuentes alternativas en el departamento de Nariño”, 
 proyecto que realiza un estudio de diversas fuentes de energía presentes en el departamento.

 La investigación comprende descargar, procesar y analizar 11 años de imágenes satelitales 
 brindadas por el sensor MODIS y 15 años de imágenes satelitales 
 brindadas por el sensor Landsat, almacenar en una base de datos las bandas necesarias para 
 la identificación de radiación solar, aplicar modelos para la construcción de mapas de radiación solar, series de tiempo de radiación solar y nubosidad, realizar un análisis enfocado 
 en la detección de patrones, análisis del comportamiento de la 
 radiación solar y nubosidad en el departamento de Nariño.

Adicionalmente se pretende establecer y compartir los diferentes tipos de información y 
metodologías aplicadas a los interesados en la temática enfocada en el procesamiento y 
análisis de imágenes satelitales MODIS y LandSat.\\

\textbf{Palabras clave:}  MODIS, LandSat, detección de patrones, detección de nubes, análisis de regresión,
 series de tiempo, patrones secuenciales.
\end{abstract}


\selectlanguage{USenglish}

\begin{abstract}
Currently the Revolution energy sources and the rapid changes of weather 
conditions forces nations to propose a change to the approaches of power generation,
for this series of changes required an analysis of each and every one of the sources of renewable energy;
in some situations it is relevant to note the conditions to optimize use of the energy potential.
Current research focuses on the construction of a time series of solar radiation 
present in the department of Nariño, with the main objective of supporting indicators
in the study of solar radiation in the “Análisis de oportunidades energéticas con 
 fuentes alternativas en el departamento de Nariño” project; This project is conducting a study of 
various sources of energy present in the department.\\

The research includes download, process and analyze 11 years of satellite images provided by MODIS sensor and 15 years of 
satellite images provided by Landsat sensor,
storing in a database the necessary bands for detecting solar radiation, making a focused pattern detection analysis,
to visualize the behavior of solar radiation on the department of Nariño.\\

Additionally it is to establish and share different types of information and
methodologies applied to those interested in the topic focused on the processing and analysis
of MODIS satellite image.


\textbf{Keywords:} MODIS, LandSat,, pattern detection, detection of clouds, regression analysis,
Time series, sequential patterns.
\end{abstract}
 

\selectlanguage{spanish}
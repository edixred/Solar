\chapter*{INTRODUCCIÓN}
\addcontentsline{toc}{chapter}{INTRODUCCIÓN}
 
Países como Alemania, España, Japón, Estado Unidos, Italia son las potencias mundiales 
en la generación de energía renovable mediante el uso de energía solar,  este tipo de 
fuente de energía limpia ha adquirido gran importancia en el entorno investigativo en los 
últimos años y han surgido grandes expectativas para el futuro. 

Cada año el sol arroja 4000 veces más energía que la que se consume, lo que demuestra 
que esta fuente energética está aún infravalorada y desaprovechada  en relación a sus potenciales energéticos; 
la implementación para esta fuente energética es posible si un territorio cuenta una buena posición 
geográfica, economía sostenible y fenómenos climáticos regulados, estas condiciones favorables para la energía a base 
de paneles solares o energía térmica se encuentran presentes en muchas regiones de América Central y América del Sur,
 como ventaja adicional la energía solar puede aprovecharse directamente o almacenarse para un consumo posterior y no tiene riesgo de agotarse.

A pesar de que el Colombia cuenta con las condiciones ideales para producir energía solar, 
aún existen regiones del territorio colombiano donde la sociedad aún esta destinada a vivir sin energía eléctrica, un caso 
particular es el territorio nariñense donde varias poblaciones costeras cuentan con la radiación solar
necesaria para abastecer el consumo requerido en esa región y adicionalmente vender la energía sobrante a poblaciones cercanas, y sin embargo
no se ha realizado un estudio enfocado en el aprovechamiento de la energía a base de radiación solar presente en el departamento de Nariño; es necesario un
estudio para el comportamiento de la radiación solar en lugares con potencial energético, un análisis del comportamiento de fenómenos 
climáticos en zonas del departamento de Nariño, esta y más información que es de gran importancia para la implementación 
de generadores a base de energía solar.

\textbf{Planteamiento del problema}

El clima colombiano posee grandes características para implementación de plantas de energía renovable debido a 
la ubicación geográfica en la zona ecuatorial, características geográficas que favorecen la presencia del clima tropical 
y una temperatura uniforme la mayor parte del año. La posición estratégica de Colombia en la zona tropical genera 
muchos beneficios a la gran mayoría de departamentos, ocasionando que los territorios reciban una mayor proporción 
de energía que el sol le transfiere a la Tierra. 

El departamento de Nariño hereda la diversidad de climas presentes en el territorio colombiano, en esta zona el clima es más 
estable y menos variable respecto a cambios en los fenómenos climáticos que obedecen a las posición de una zona o altitud 
donde se encuentra un lugar determinado, la información referente al comportamiento de estas característica son de gran 
beneficio para aprovechar de forma óptima las diferentes fuentes de energía renovable. El proyecto “Análisis de oportunidades energéticas con fuentes alternativas 
en el departamento de Nariño” busca aprovechar los recursos naturales disponibles en este departamento con el objetivo inicial de
estudiar y analizar las fuentes de energías biomasa, hidráulica, viento y solar que se encuentran presentes dentro del departamento; 
el presente proyecto de investigación busca realizar el estudio previo para identificar las zonas con posible potencial solar para generar 
energía, la radiación solar es una fuente renovable utilizada mucho tiempo atrás por paises desarrollados, y actualmente  el manejo de esta 
fuente energética es una tendencia de las grandes potencias mundiales para mejorar la economía y mitigar los efectos de agentes contaminates.

La información que provee el IDEAM no permite tener gran certeza en los datos de radiación para una zona determinada, esto se debe a que cuenta con 
un número reducido de estaciones y es imposible tratar de generalizar datos aproximados para un territorio con grandes variaciones en su terreno, por este motivo
se puede decir que en la actualidad el departamento de Nariño no cuenta con un estudio preciso en información de radiación solar,información que podría 
ser usada y aprovechada para plantas energéticas a base de las fuentes de energía renovable. Tener un registro histórico de 
la información climatológica es muy importante debido a que se puede hacer una predicción del comportamiento de fenómenos climáticos como nubes, viento, 
características de la vegetación, radiación solar, precipitaciones o datos que permitan la identificación de zonas con potencial 
energético en una zona específica del territorio nariñense. El actual proyecto de investigación se enfoca 
en el estudio de radiación solar, mediante la construcción de una serie de tiempo de 11 años de información referente a 
radiación solar, comportamiento de la radiación en zonas determinadas y el comportamiento de nubes sobre el 
departamento de Nariño; actualmente la información referente a la cantidad de energía radiante que llega a la superficie del 
departamento en una época o tiempo determinado es difícil establecer y muy complejo de predecir; por este motivo el presente 
estudio resuelve esta carencia de información, la investigación radica en el procesamiento y análisis de imágenes 
satelitales  de libre acceso proveídas por el sensor MODIS y Landsat, imágenes que a partir de las características y propiedades
del sensor presenta determinado número de bandas con la capacidad de representar la medida de la gama de frecuencias capturada 
por el sensor, esta información representada en un formato determinado permite establecer características geográficas, hidrográficas, 
climáticas, vegetación, gases atmosféricos, radiación solar, precipitación y otro tipo información adicional del planeta
que se encuentra presentes en los últimos sensores activos. 

El estudio de energía solar mediante procesamiento y análisis de imágenes satelitales actualmente está 
apoyado por el uso de herramientas de libre acceso especializadas para realizar análisis de la información, aplicación de modelos
de regresión, construcción de series de tiempo, visualización del potencial energético a través de mapas de radiación solar 
y detección de patrones secuenciales en una serie de tiempo.

\textbf{Objetivo general}

Detectar patrones secuenciales que permitan predecir el comportamiento de fenómenos climáticos y su impacto en  áreas con potencial 
de radiación solar dentro del departamento de Nariño.


\textbf{Objetivos Específicos}

\begin{itemize}
 \item Construir series de tiempo de radiación solar dentro del área de estudio a partir del procesamiento de imágenes satelitales de libre acceso.
 \item Detectar la presencia de nubosidad en las imágenes satelitales obtenidas utilizando los algoritmos más pertinentes.
 \item Seleccionar y aplicar herramientas para la detección de patrones secuenciales.
 \item Generar un conjunto de datos que relacione la presencia de nubosidad en regiones con potencial de radiación solar.
 \item Validar la relación de resultados obtenido mediante la comparación datos producto  del estudio con datos reales capturados con algunos sensores activos.
 \item Generar mapas y reportes que permitan visualizar la radiación solar presente en el departamento de Nariño.
\end{itemize}


\textbf{Organización del documento}

Este documento está organizado en 5 secciones, la primera sección abarca los fundamentos teóricos empleados en la investigación,
la sección 2 presenta la metodología utilizada para el desarrollo de la investigación, esta sección se desarrolla en las 
siguientes fases: a)la identificación y adquisición de imágenes satelitales, b)procesamiento y almacenamiento de la 
información, c)selección, aplicación del modelo de regresión y construcción de series de tiempo, d)mapas de radiación solar usando 
interpolación espacial, e)detección de patrones secuenciales, f)comportamiento de las nubes; la sección 3 contiene la información 
referente a la validación de la información, en la sección 4 se presentan las conclusiones y trabajos futuros 
y para finalizar se presenta referencias bibliográficas y los anexos correspondientes.